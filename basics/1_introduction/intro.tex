\documentclass[aspectratio=169]{beamer}
\usepackage[utf8]{inputenc}
\usetheme{pmos_modified}

\title{postmarketOS}
\subtitle{Introduction}
\author{Piotr Halama}
\date{\today}

\begin{document}

\frame{\titlepage}

\begin{frame}{What are the problems with Android?}
\begin{itemize}
	\item Every port is a new fork of a system and kernel
	\begin{itemize}
		\item hard to update kernel
		\item fixing one thing requires to change all affected repos
		\item hacky solutions for hardware drivers
	\end{itemize}
	\item No updates after year or two
	\item Rather bloated, requires many hardware resources
	\item proprietary bits
	\begin{itemize}
		\item may contain bugs or backdoors, very hard to check
		\item impossible to update
		\item hard to replace
		\item may work only with one specific kernel version
	\end{itemize}
\end{itemize}
\end{frame}

\begin{frame}{What is postmarketOS?}
	\center \Large Linux distribution for smartphones
	\center based on Alpine Linux
\end{frame}	

\begin{frame}{Goals of postmarketOS}
\begin{itemize}
	\item Few device-specific packages
	\begin{itemize}
		\item device package
		\item downstream kernel package
		\item firmware package
	\end{itemize}
	\item Porting mainline kernel to devices
	\begin{itemize}
		\item everything is standarized
		\item easy to update
		\item one kernel for many devices
	\end{itemize}
	\item 10 year life-cycle
	\begin{itemize}
		\item Device gets updates as long as it can handle them
		\item some devices (e.g. Nokia N900) already passed 10 years and it's still supported
	\end{itemize}
%	\item Have fun
%	\begin{itemize}
%		\item You can work on anything you want
%	\end{itemize}
	\item Document everything
%	\begin{itemize}
%		%\item Have fun
%		\item You can work on anything you want
%		\item Note everything in wiki, it may come in handy
%	\end{itemize}
	
\end{itemize}
\end{frame}

\begin{frame}{Pros}
\begin{itemize}
	\item Can run on devices stuck with Android 2.3.6
	\item Faster builds than Android
	\item pmbootstrap takes care of building enviroment
	\begin{itemize}
		\item building is done in chroots, no need to install additional stuff or worry about buildchain
		%TODO is this really important ?
		%\item smaller size, repo contains just build instructions and patches, heavy source code is downloaded per package if needed
	\end{itemize}
\end{itemize}
\end{frame}

\begin{frame}{Cons}
\begin{itemize}
	\item Still in early development
	\begin{itemize}
		\item calls don't work yet on most devices
		\item many phones have only basic port with only screen and USB networking working 
		\item upstream changes in Alpine can break already working pieces
	\end{itemize}
	\item Software selection is slim
	\begin{itemize}
		\item many programs are not optimized for touchscreen
	\end{itemize}
	\item Security is not top priority (for now)
%	\begin{itemize}
%		\item DEs may run as root
%		\item No privilege separation
%	\end{itemize}
\end{itemize}
\end{frame}

\begin{frame}{We are not alone}
\begin{itemize}
	\item Desktop Enviroments
	\begin{itemize}
	\item KDE Plasma, XFCE, touch and keyboard/mouse optimized DE's
	\end{itemize}
	\item Halium and libhybris
	\begin{itemize}
		\item Android in a container and translation layer allows to run proprietary drivers
	\end{itemize}
	\item Other Linux distros for smart devices
	\begin{itemize}
	\item AsteroidOS
	\item LuneOS
	\item Maemo Leste
	\item Sailfish OS
	\item SHR
	\item UBports
	\end{itemize}
	\item Hardware makers
	\begin{itemize}
		\item Librem 5
		\item PinePhone and PineTab
	\end{itemize}
\end{itemize}
\end{frame}

\begin{frame}{}%Raw numbers?}
	\center \Large Thanks for watching!
\end{frame}

%TODO remove this in production!
\begin{frame}{To do in voiceover only (and captions)}
\begin{itemize}
\item "Document everything" -  mention wiki, GH/GL issues, knowledge exchange with others
	\item "work on anything you want" - basically have fun
	\item "in early development" - rather for hackers and tinkerers
	\item version based on stable Alpine is coming
	\item libre drivers
\end{itemize}
\end{frame}

\end{document}